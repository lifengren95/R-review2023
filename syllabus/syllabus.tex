\documentclass{article}
\usepackage[margin=1in]{geometry}
\usepackage[utf8]{inputenc}
\usepackage[english]{babel}
\usepackage[]{amsthm} %lets us use \begin{proof}
\usepackage[]{amssymb} %gives us the character \varnothing
\usepackage{enumerate}
\usepackage{nopageno}
\usepackage{amsfonts}
\usepackage{fancyhdr}
\usepackage{comment}
\usepackage{times}
\usepackage{amsmath}
\usepackage{physics}
\usepackage{changepage}
\usepackage{graphicx}
\usepackage{subfigure}
\setcounter{MaxMatrixCols}{30}
\usepackage{multirow}
\usepackage{caption}
\usepackage{subcaption}
\usepackage{booktabs}
\usepackage{bm}
\usepackage{mdframed}
\usepackage{tcolorbox}
\newcommand{\rd}[1]{{\textcolor{red}{#1}}}
\newcommand{\ph}{\phantom{-}}
\newcommand{\pp}{\phantom{-1}}
\newcommand{\indep}{\perp \!\!\! \perp}
\usepackage[T1]{fontenc}
\usepackage{textcomp}
\usepackage{mathpazo}
\usepackage[framed,numbered,autolinebreaks,useliterate]{mcode}

%\usepackage[colorlinks=true, linkcolor=black, citecolor=blue, urlcolor=blue]{hyperref}
%\usepackage{hyperref}       % This one with red boxes arounded, the above one does not
\usepackage{url}            % simple URL typesetting
\newtheorem{definition}{Definition}
\usepackage{nicefrac}       % compact symbols for 1/2, etc.
\usepackage{microtype}      % microtypography
%\usepackage{lipsum}		% Can be removed after putting your text content
%\usepackage{natbib}
%\usepackage{doi}
%\setcitestyle{aysep={,}}
\newtcolorbox{theo}{colback=red!5!white,colframe=red!75!black,left=5pt,right=5pt,top=2pt,bottom=2pt,}

\newtcolorbox{tip}{colback=blue!5!white,colframe=blue!75!black,left=5pt,right=5pt,top=2pt,bottom=2pt,}
\usepackage{titlesec}
\usepackage{tabularx}
\titleformat*{\section}{\large\bfseries}
\usepackage[hidelinks]{hyperref}
\usepackage{xcolor}
\hypersetup{
    colorlinks,
    linkcolor={blue!50!black},
    citecolor={blue!50!black},
    urlcolor={blue!80!black}
}

\pagestyle {fancy}
\title{APEC R Review 2023 Summer}
\author{Instructor: Lifeng Ren}
\date{August 14, 2023 to August 18, 2023}
\lhead{R Review-Syllabus}
\chead{Lifeng Ren}
\rhead{2023 Summer}


\begin{document}
\thispagestyle{plain}
\begin{center}
{\LARGE \textsc{Introduction to R Statistical Analysis Software}}
\end{center}
\begin{center}
\large \textsc{Syllabus}\\
Summer 2023(August 14, 2023 - August 18, 2023)
\end{center}
%\date{September 26, 2014}

\begin{center}
\rule{6.5in}{0.4pt}
\begin{tabular}{llll}
\textbf{Instructor:} & Lifeng Ren &   \textbf{Time:} & 1PM-4PM, Central Time \\
\textbf{Email:} &\href{mailto:ren00154@umn.edu}{ren00154@umn.edu}  &   \textbf{Place:} & Ruttan 135B and \href{https://umn.zoom.us/j/96218181588?pwd=ZFVrdWQ0bkZyb2N2Wkdwc0JnUk9CQT09}{Zoom}\\
\textbf{Office Hours:} & After class: 4PM-5PM; Ruttan 135B\& \href{https://umn.zoom.us/j/96218181588?pwd=ZFVrdWQ0bkZyb2N2Wkdwc0JnUk9CQT09}{Zoom}&
\end{tabular}
\rule{6.5in}{0.4pt}
\end{center}

\section*{Course Description}
This course is a preliminary introduction to the R statistical software, specifically tailored for new graduate students. 
All the lecture materials will be provided on the Github Repository (\href{https://github.com/lfr00154/R-review2023}{Here:To be Published Later}), and on Canvas. 
\begin{itemize}
    \item Rather than encompassing every aspect of R, the course aims to provide a strong groundwork for Econometric Analysis (APEC8211- 8212). 
    \item Programming for Econometrics (APEC8221), and Big Data Methods in Economics (APEC8222) are the classes you might want to register if you want to learn more about coding in R and Python. 
    \item The teaching goal of this class is to provide everyone with solid R-programming foundation for the first-year PhD Econometrics Class. After this class, you should be able to:
    \begin{itemize}
        \item Code in IDE (R-studio) with both R-scripts, and R-markdown.
        \item Understand the data types and how to save and acess to the data in R.
        \item Obtain basic data cleaning and manupulations skills in R.
        \item Write simple functions with loops in R.
        \item Run regressions in R.
        \item Visulize basic graphs in R. 
    \end{itemize}
\end{itemize}

\section*{Before Class}
\begin{itemize}
    \item Download and Install R and R studio on your desktop from this \href{https://posit.co/download/rstudio-desktop/}{website} 
    \item Get your UCard access to Ruttan:
    \begin{itemize}
        \item \href{https://ucard.tc.umn.edu/your-u-card/getting-your-first-u-card}{Get a U card} so you can access the building. 
        \item Request advanced access to the building (Ruttan Hall): \href{https://docs.google.com/forms/d/e/1FAIpQLScTSUXcENXbf5PBaqZPNW_kaDG2gEaVljUVd8saW4M4IFY3DQ/viewform}{Ruttan Hall Access Request Form}. 
        \item If you have questions, reach out to Melissa Isle (webe0342@umn.edu)
    \end{itemize}
    \item Finish the \href{https://forms.gle/rqeCeCi6PYFNpSbw6}{Survey}.
    \item Bring your laptop to the class. 

\end{itemize}


\section*{Reference}
There are many excellent references out there. Here are some textbooks or online available notes I suggest you guys read or at least have a PDF version in hand for reference. 

\begin{itemize}
    \item \href{https://r4ds.had.co.nz/index.html}{R for Data Science} (Personally recommended)
    \item \href{https://jrnold.github.io/r4ds-exercise-solutions/}{Solutions for the book: R for Data Science}
    \item \href{https://www.econometrics-with-r.org/index.html}{Introduction to Econometrics with R}
    \item \href{https://libguides.bates.edu/r/r-for-economics}{R for Economics}
    \item \href{https://rkabacoff.github.io/datavis/datavis.pdf}{Data Visualization with R: by Rob Kabacoff}
    \item Matloff, Norman. The art of R programming: A tour of statistical software design. No Starch Press, 2011. 
    \begin{itemize}
        \item This is the one that Programming for Econometrics Class will be using and mostly used for previous R Review Class.
        \item PDF version available through UMN Library or you can find it online
    \end{itemize}
\end{itemize}

\subsection*{Syntax Cheat Sheet}
\begin{itemize}
    \item \hyperlink{https://www.rstudio.com/wp-content/uploads/2015/02/data-wrangling-cheatsheet.pdf}{Data Wrangling
with dplyr and tidyr}
    \item \href{https://web.mit.edu/hackl/www/lab/turkshop/slides/r-cheatsheet.pdf}{Basic}
    \item \href{https://www.beoptimized.be/pdf/R_Data_Transformation.pdf}{data.table} 
    \item \href{https://nyu-cdsc.github.io/learningr/assets/data-transformation.pdf}{dplyr}
\end{itemize}

\section*{Exercise}
To learning coding better, we all need practice more. So, we will have several exercises both in-class and after-class. They are \textbf{optional} and are only designed to help you improve your level of understanding. 
\begin{itemize}
    \item I will provide a typed answer key to all exercise I assigned with R Markdown (.rmd) code. You are free to use my code as a template for yourself.
    \item My teaching goal is to make sure everyone can finish the in-class exercises. If possible, we will spend couple minitues every class to go over the after-class exercise as a review first.   
    
    \end{itemize}


\section*{Class Style}
We will code together and I will expalain the code while we are coding them. We will have a 5 mintues break every hour. 

\section*{Tentative Class Schedule}

\begin{tabularx}{\textwidth}{l|X}
\toprule
Date: & Tasks \\\midrule\\
Before Class: & Install the R, R studio and required libabry\\\\
8/14/2023     &      Workflow; Other necessary coding/study tools for PhD studies. Introduction to R: dataframe and R-studio interface. \\\\
8/15/2023                & Data Manupulation and Data Cleaning: tidyverse, data.table\\\\
8/16/2023          & Functions, loops, and Simulations. \\\\
8/17/2022               &   Econometrics with R:aer \\\\
8/18/2022               & Data Visulization: ggplot2\\
\bottomrule
\end{tabularx}







\end{document}

